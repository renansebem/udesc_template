\documentclass[a4paper,12pt]{udesc}

\usepackage{tabularx}


%% Declaração de Símbolos
%\DeclareAcronym{alpha}{
%  short = $\alpha$ ,
%  long  = Alfa ,
%  class = symbol
%} 
%\DeclareAcronym{beta}{
%  short = $\beta$ ,
%  long  = Beta ,
%  class = symbol
%} 
%
%
%% Declaração de Abreviações
%\DeclareAcronym{ny}{
%  short = NY ,
%  long  = New York ,
%  class = abbrev
%} 
%\DeclareAcronym{abnt}{
%  short = ABNT ,
%  long  = Associação Brasileira de Normas Técnicas ,
%  class = abbrev
%}


%\let\su@ExpandTwoArgs\relax 
%\let\IfSubStringInString\relax 
%\let\su@IfSubStringInString\relax 


\usepackage[acronym,symbols,nopostdot]{glossaries}

\renewcommand{\glsnamefont}[1]{\textbf{#1}}

\makenoidxglossaries

\setacronymstyle{long-short}
\newacronym{svm}{SVM}{support vector machine support vector machine support vector machine support vector machine support vector machine support vector machine support vector machine support vector machine support vector machine }
\newacronym{svmmm}{SVq}{support vector machine}




\newglossaryentry{beta}
{
name={\ensuremath{\beta}},
sort={beta},
description={the set containing no elements},
type=symbols
}



\newglossaryentry{alfa}
{
name={aallffaa},
sort={alfa},
description={asasasasasasasas},
type=symbols
}


%\newsymbol{beta}{$\beta$}{Beta}

\usepackage[alf,abnt-emphasize=bf]{abntex2cite}		% Referências padrão ABNT

\usepackage{hyphenat}


\begin{document}

\selectlanguage{brazil}

\autor{Nome do Autor}

\titulo{Título do trabalho}
\subtitulo{Subtítulo}

\tipo{1} %Tese de Doutorado
%\tipo{2} %Dissertação de Mestrado
%\tipo{3} %Trabalho de Conclusão de Curso
%\tipo{4} %Relatório de Estágio
%\tipo{5} %Relatório de Pós Doutorado
%\tipo{6} %Trabalho de Conclusão de Curso de Especialização


\orientador{Nome do Orientador}

\coorientador{Nome do Co-orientador}

\comentario{Trabalho de Conclusão de Curso, apresentado como requisito parcial para obtenção do Grau de Bacharel em Engenharia Elétrica pela Universidade do Estado de Santa Catarina, no Centro de Ciências Tecnológicas.}

\instituicao{
\MakeUppercase{Universidade do Estado de Santa Catarina - UDESC}\par 
\MakeUppercase{Centro de Ciências Tecnológicas - CCT}\par 
\MakeUppercase{Departamento de Engenharia Elétrica - DEE} }

\local{Joinville, SC}

\ano{2018}

\capa

\folhaderosto
%%%%%%%%%%%%%%%%%%%%%%%%%%%%%%%%%%%%%%%%%%%%%%%%%%%%%%%%%%%%%%%%%%%%%%%%%%%%%%%%%%%%
% FICHA CATALOGRAFICA
%%%%%%%%%%%%%%%%%%%%%%%%%%%%%%%%%%%%%%%%%%%%%%%%%%%%%%%%%%%%%%%%%%%%%%%%%%%%%%%%%%%%

%\fichacatalografica


%\newcommand{\ficha_catalografica}%
%{%
\vspace*{9.75cm}%
\begin{small}%
\noindent%
\begin{tabularx}{\textwidth}{|lX|}%
\hline%
SXXXc      &      Sobrenome, Nome \\%
                  &       \hspace{0.5cm}     \titulodatanormal\subtitulodatanormal/ \autordatanormal. – \anodata.\\%
                            &  \hspace{0.5cm}   xxx p.  :  il. ; XX cm  \\%
 & \\%
                   &       \hspace{0.5cm}     Orientador: \orientadordata \\%
                    &    \hspace{0.5cm}       Bibliografia: xxx-xxx \\%
                     &   \hspace{0.5cm}       \tipodatacitacao \ –  Universidade do Estado de Santa Catarina, Centro de
                         Ciências Tecnológicas, Programa de Pós-Graduação em XXX, Joinville, \anodata. \\
  &       \\%    
                      &  \hspace{0.5cm}       1. Palavras. 2. Chave. I. \orientadordata.  II. Universidade do 
                        Estado de Santa Catarina.  Programa de Pós-Graduação em XXX. III.
                       Título.                      \\
   &       \\%                          
                     &                                                             \hfill                                                CDD XXX.X –  XX. ed. \\%
\hline%
\end{tabularx}%
\end{small}%
\pagebreak%
%}

%Folha de arovação sem assinatura:
%%%%%%%%%%%%%%%%%%%%%%%%%%%%%%%%%%%%%%%%%%%%%%%%%%%%%%%%%%%
% FOLHA DE APROVACAO TIPO 1
%%%%%%%%%%%%%%%%%%%%%%%%%%%%%%%%%%%%%%%%%%%%%%%%%%%%%%%%%%%

\begin{folhadeaprovacao}

Trabalho de Conclusão de Curso, apresentado como requisito parcial para obtenção do Grau de Bacharel em Engenharia Elétrica pela Universidade do Estado de Santa Catarina, no Centro de Ciências Tecnológicas, avaliada pela banca examinadora constituída pelos
professores: \setlength{\signthickness}{0.4pt}


\assinaturaor{Prof. Dr. Nome do Orientador\\ Universidade do Estado de Santa Catarina}

\assinaturacoor{Prof. Dr. Nome do Co-orientador\\ Universidade do Estado de Santa Catarina}




\assinaturamembrotipoum{Prof. Dr. Nome do Membro \\ Universidade do Estado de Santa Catarina}
%\assinaturamembrotipoum{Prof. Dr. Nome do Membro \\ Universidade do Estado de Santa Catarina}
\assinaturamembrotipoum{Prof\textsuperscript{a}. Dr\textsuperscript{a}. Nome do Membro \\ Universidade do Estado de Santa Catarina}


\data{24 de junho de 2014}
\end{folhadeaprovacao}


%%%%%%%%%%%%%%%%%%%%%%%%%%%%%%%%%%%%%%%%%%%%%%%%%%%%%%%%%%%
% FOLHA DE APROVACAO TIPO 2
%%%%%%%%%%%%%%%%%%%%%%%%%%%%%%%%%%%%%%%%%%%%%%%%%%%%%%%%%%%

\begin{folhadeaprovacao}

Trabalho de Conclusão de Curso, apresentado como requisito parcial para obtenção do Grau de Bacharel em Engenharia Elétrica pela Universidade do Estado de Santa Catarina, no Centro de Ciências Tecnológicas, avaliada pela banca examinadora constituída pelos
professores: \setlength{\signthickness}{0.4pt}
%\quantidadeass{5}

\assinaturaor{Prof. Dr. Nome do Orientador\\ Universidade do Estado de Santa Catarina}

\assinaturacoor{Prof. Dr. Nome do Orientador\\ Universidade do Estado de Santa Catarina}

\vfill
Membros:


\begin{minipage}{\textwidth/2-1cm}
\assinaturamembrotipodoisAquatro{Prof. Dr. Nome do Membro \\ Universidade do Estado de Santa Catarina}

\assinaturamembrotipodoisAquatro{Prof. Dr. Nome do Membro \\ Universidade do Estado de Santa Catarina}
\end{minipage}\hspace{2cm}\begin{minipage}{\textwidth/2-1cm}
\assinaturamembrotipodoisAquatro{Prof. Dr. Nome do Membro \\ Universidade do Estado de Santa Catarina}

\assinaturamembrotipodoisAquatro{Prof\textsuperscript{a}. Dr\textsuperscript{a}. Nome do Membro \\ Universidade do Estado de Santa Catarina}
\end{minipage}

\data{24 de junho de 2014}
\end{folhadeaprovacao}

%Quando a folha de aprovacao for assinada ela pode ser escaneada em pdf e adicionada aqui:
%\includepdf[pages=1]{pre-textual/folhaaprova.pdf}
\input{pre-textual/ded-agrad}


%%%%%%%%%%%%%%%%%%%%%%%%%%%%%%
% RESUMO
%%%%%%%%%%%%%%%%%%%%%%%%%%%%%%

% 150 a 500 palavras

% Palavras-chave: de 3 a 5.


\begin{resumo}

Escreva aqui o resumo. Parágrafo Único.

\textbf{Palavras-chave:}  Entre Três. E Cinco. Palavras-chave.

\end{resumo}

\begin{udescabstract}
\selectlanguage{english}

Write abstract here.

\textbf{Keywords:} Between Three. And Five. Keywords.
\selectlanguage{brazil}
\end{udescabstract}
\pagestyle{vazio}
\cleardoublepage
\listoffigures
\clearpage

\cleardoublepage
\listoftables
\clearpage

\cleardoublepage

\cleardoublepage
\listofdefcaption
\clearpage
\cleardoublepage


\printnoidxglossary[type=acronym,style=super,nonumberlist,title=Lista de Abreviaturas,toctitle=Lista de Abreviaturas]% sem n da página
%\printnoidxglossary[type=acronym,style=super]% com n da página


%\printacronyms[include-classes=abbrev,heading = chapter*, name=Lista de Abreviaturas e Siglas]
\thispagestyle{empty}

\clearpage

\cleardoublepage
\pagestyle{vazio}
\printnoidxglossary[type=symbols,style=super,nonumberlist,title=Lista de Símbolos, toctitle=Lista de Símbolos]% sem n da página
%\printacronyms[include-classes=symbol,heading = chapter*, name= Lista de Símbolos]
\thispagestyle{empty}

\clearpage

\cleardoublepage

\tableofcontents

\clearpage

\cleardoublepage

\pagestyle{udesc}



\chapter{Versão atualizada do template}

É possível baixar a versão atualizada deste template em:


\begin{verbatim}
https://drive.google.com/#folders/0B9qjhr0sGTJeX3c3dTUyQXlGREk
\end{verbatim}

\definicao{item da lista customizada}{bla}

\chapter{Símbolos e Abreviações}

Declarar abreviações e símbolos no preambulo.


First use: \gls{svm}. Second use: \gls{svm}.

First use: \gls{svmmm}. Second use: \gls{svmmm}.



\gls{beta}


%Abreviações: \ac{ny}, \ac{abnt}.



%Símbolos:  \aca{beta} = \aca{alpha}.


%Nas demais vez que usar a abreviação a descrição não aparecerá: \ac{abnt}.

\chapter{Citações e Referências}

Exemplo de citações: \cite{RW89}.

Também: \cite{CASS99}. 

Citação em linha \citeonline{CASS99}.


          Do mesmo modo, a percepção das dificuldades nos obriga à análise do investimento em reciclagem técnica. Por outro lado, a execução dos pontos do programa cumpre um papel essencial na formulação da gestão inovadora da qual fazemos parte. Todavia, o desenvolvimento contínuo de distintas formas de atuação causa impacto indireto na reavaliação dos modos de operação convencionais. Evidentemente, a determinação clara de objetivos talvez venha a ressaltar a relatividade das posturas dos órgãos dirigentes com relação às suas atribuições.




          A certificação de metodologias que nos auxiliam a lidar com o julgamento imparcial das eventualidades exige a precisão e a definição do retorno esperado a longo prazo. No mundo atual, o fenômeno da Internet apresenta tendências no sentido de aprovar a manutenção das direções preferenciais no sentido do progresso. Pensando mais a longo prazo, o consenso sobre a necessidade de qualificação obstaculiza a apreciação da importância dos paradigmas corporativos. As experiências acumuladas demonstram que a estrutura atual da organização auxilia a preparação e a composição dos métodos utilizados na avaliação de resultados.
          
          
          
          Do mesmo modo, a percepção das dificuldades nos obriga à análise do investimento em reciclagem técnica. Por outro lado, a execução dos pontos do programa cumpre um papel essencial na formulação da gestão inovadora da qual fazemos parte. Todavia, o desenvolvimento contínuo de distintas formas de atuação causa impacto indireto na reavaliação dos modos de operação convencionais. Evidentemente, a determinação clara de objetivos talvez venha a ressaltar a relatividade das posturas dos órgãos dirigentes com relação às suas atribuições.




          A certificação de metodologias que nos auxiliam a lidar com o julgamento imparcial das eventualidades exige a precisão e a definição do retorno esperado a longo prazo. No mundo atual, o fenômeno da Internet apresenta tendências no sentido de aprovar a manutenção das direções preferenciais no sentido do progresso. Pensando mais a longo prazo, o consenso sobre a necessidade de qualificação obstaculiza a apreciação da importância dos paradigmas corporativos. As experiências acumuladas demonstram que a estrutura atual da organização auxilia a preparação e a composição dos métodos utilizados na avaliação de resultados.



          Do mesmo modo, a percepção das dificuldades nos obriga à análise do investimento em reciclagem técnica. Por outro lado, a execução dos pontos do programa cumpre um papel essencial na formulação da gestão inovadora da qual fazemos parte. Todavia, o desenvolvimento contínuo de distintas formas de atuação causa impacto indireto na reavaliação dos modos de operação convencionais. Evidentemente, a determinação clara de objetivos talvez venha a ressaltar a relatividade das posturas dos órgãos dirigentes com relação às suas atribuições.




          A certificação de metodologias que nos auxiliam a lidar com o julgamento imparcial das eventualidades exige a precisão e a definição do retorno esperado a longo prazo. No mundo atual, o fenômeno da Internet apresenta tendências no sentido de aprovar a manutenção das direções preferenciais no sentido do progresso. Pensando mais a longo prazo, o consenso sobre a necessidade de qualificação obstaculiza a apreciação da importância dos paradigmas corporativos. As experiências acumuladas demonstram que a estrutura atual da organização auxilia a preparação e a composição dos métodos utilizados na avaliação de resultados.



          Do mesmo modo, a percepção das dificuldades nos obriga à análise do investimento em reciclagem técnica. Por outro lado, a execução dos pontos do programa cumpre um papel essencial na formulação da gestão inovadora da qual fazemos parte. Todavia, o desenvolvimento contínuo de distintas formas de atuação causa impacto indireto na reavaliação dos modos de operação convencionais. Evidentemente, a determinação clara de objetivos talvez venha a ressaltar a relatividade das posturas dos órgãos dirigentes com relação às suas atribuições.




          A certificação de metodologias que nos auxiliam a lidar com o julgamento imparcial das eventualidades exige a precisão e a definição do retorno esperado a longo prazo. No mundo atual, o fenômeno da Internet apresenta tendências no sentido de aprovar a manutenção das direções preferenciais no sentido do progresso. Pensando mais a longo prazo, o consenso sobre a necessidade de qualificação obstaculiza a apreciação da importância dos paradigmas corporativos. As experiências acumuladas demonstram que a estrutura atual da organização auxilia a preparação e a composição dos métodos utilizados na avaliação de resultados.



          Do mesmo modo, a percepção das dificuldades nos obriga à análise do investimento em reciclagem técnica. Por outro lado, a execução dos pontos do programa cumpre um papel essencial na formulação da gestão inovadora da qual fazemos parte. Todavia, o desenvolvimento contínuo de distintas formas de atuação causa impacto indireto na reavaliação dos modos de operação convencionais. Evidentemente, a determinação clara de objetivos talvez venha a ressaltar a relatividade das posturas dos órgãos dirigentes com relação às suas atribuições.




          A certificação de metodologias que nos auxiliam a lidar com o julgamento imparcial das eventualidades exige a precisão e a definição do retorno esperado a longo prazo. No mundo atual, o fenômeno da Internet apresenta tendências no sentido de aprovar a manutenção das direções preferenciais no sentido do progresso. Pensando mais a longo prazo, o consenso sobre a necessidade de qualificação obstaculiza a apreciação da importância dos paradigmas corporativos. As experiências acumuladas demonstram que a estrutura atual da organização auxilia a preparação e a composição dos métodos utilizados na avaliação de resultados.


\section{Section}



First use: \gls{svm}. Second use: \gls{svm}.

First use: \gls{svmmm}. Second use: \gls{svmmm}.


\glssymbol{beta}

\glssymbol{beta}

\glssymbol{beta}

Acima de tudo, é fundamental ressaltar que o entendimento das metas propostas assume importantes posições no estabelecimento dos índices pretendidos. Neste sentido, a competitividade nas transações comerciais acarreta um processo de reformulação e modernização das formas de ação. Ainda assim, existem dúvidas a respeito de como a valorização de fatores subjetivos oferece uma interessante oportunidade para verificação do processo de comunicação como um todo. Todas estas questões, devidamente ponderadas, levantam dúvidas sobre se a consulta aos diversos militantes deve passar por modificações independentemente do orçamento setorial. No entanto, não podemos esquecer que a hegemonia do ambiente político agrega valor ao estabelecimento dos níveis de motivação departamental. 

          O que temos que ter sempre em mente é que a necessidade de renovação processual facilita a criação das condições inegavelmente apropriadas. Nunca é demais lembrar o peso e o significado destes problemas, uma vez que a expansão dos mercados mundiais ainda não demonstrou convincentemente que vai participar na mudança do sistema de formação de quadros que corresponde às necessidades. Gostaria de enfatizar que o acompanhamento das preferências de consumo faz parte de um processo de gerenciamento de alternativas às soluções ortodoxas. O empenho em analisar a contínua expansão de nossa atividade pode nos levar a considerar a reestruturação das condições financeiras e administrativas exigidas. 


\subsection{Subsection}

          A nível organizacional, a crescente influência da mídia estende o alcance e a importância das diretrizes de desenvolvimento para o futuro. Assim mesmo, o desafiador cenário globalizado é uma das consequências de todos os recursos funcionais envolvidos. Caros amigos, o surgimento do comércio virtual promove a alavancagem das novas proposições. Podemos já vislumbrar o modo pelo qual a revolução dos costumes não pode mais se dissociar do sistema de participação geral. 

          É importante questionar o quanto o aumento do diálogo entre os diferentes setores produtivos possibilita uma melhor visão global do fluxo de informações. O incentivo ao avanço tecnológico, assim como o comprometimento entre as equipes estimula a padronização do levantamento das variáveis envolvidas. Por conseguinte, a consolidação das estruturas maximiza as possibilidades por conta das diversas correntes de pensamento. Percebemos, cada vez mais, que a adoção de políticas descentralizadoras prepara-nos para enfrentar situações atípicas decorrentes do impacto na agilidade decisória. 

\subsubsection{Subsubsection}

Citação em linha: \citeonline{JC08}.

\cite{MQ04}
\cite{lopes2012dissertacao}
\cite{CL89}

          É claro que a constante divulgação das informações garante a contribuição de um grupo importante na determinação das regras de conduta normativas. A prática cotidiana prova que o novo modelo estrutural aqui preconizado aponta para a melhoria dos procedimentos normalmente adotados. Desta maneira, a complexidade dos estudos efetuados representa uma abertura para a melhoria dos conhecimentos estratégicos para atingir a excelência. 

          O cuidado em identificar pontos críticos no início da atividade geral de formação de atitudes desafia a capacidade de equalização dos relacionamentos verticais entre as hierarquias. Não obstante, a mobilidade dos capitais internacionais afeta positivamente a correta previsão do remanejamento dos quadros funcionais. Do mesmo modo, o desenvolvimento contínuo de distintas formas de atuação nos obriga à análise do investimento em reciclagem técnica. O empenho em analisar a execução dos pontos do programa afeta positivamente a correta previsão dos relacionamentos verticais entre as hierarquias. Desta maneira, o início da atividade geral de formação de atitudes oferece uma interessante oportunidade para verificação das posturas dos órgãos dirigentes com relação às suas atribuições.

\chapter{Figuras e Tabelas}

\begin{figure}[h!]
\centering
\caption{Logo da UDESC.}

\includegraphics[scale=0.04]{fig/Marca_UDESC_vertical.pdf}


Fonte: http://www.udesc.br/?id=899


\end{figure}





\begin{table}[h!]
\caption{Tabela.}
\centering
\vspace{0.2cm}
\begin{tabular}{cc}
\hline
Cabeçalho & Dados \\
\hline

1 & 2\\
3 & 4\\
\hline


\end{tabular}

\end{table}


\chapter{bla}


\begin{figure}[h!]
\centering
\caption{Logo da UDESC.}

\includegraphics[scale=0.04]{fig/Marca_UDESC_vertical.pdf}


Fonte: \url{http://www.udesc.br/?id=899}


\end{figure}


          Evidentemente, a determinação clara de objetivos promove a alavancagem de alternativas às soluções ortodoxas. A certificação de metodologias que nos auxiliam a lidar com o julgamento imparcial das eventualidades exige a precisão e a definição das diversas correntes de pensamento. No mundo atual, a mobilidade dos capitais internacionais agrega valor ao estabelecimento dos procedimentos normalmente adotados. Ainda assim, existem dúvidas a respeito de como o fenômeno da Internet representa uma abertura para a melhoria das condições inegavelmente apropriadas. Todavia, a crescente influência da mídia assume importantes posições no estabelecimento dos métodos utilizados na avaliação de resultados. 

          Gostaria de enfatizar que a percepção das dificuldades auxilia a preparação e a composição das direções preferenciais no sentido do progresso. Pensando mais a longo prazo, a revolução dos costumes acarreta um processo de reformulação e modernização do processo de comunicação como um todo. Não obstante, a expansão dos mercados mundiais talvez venha a ressaltar a relatividade das formas de ação. 

          O incentivo ao avanço tecnológico, assim como a hegemonia do ambiente político cumpre um papel essencial na formulação dos níveis de motivação departamental. Caros amigos, a necessidade de renovação processual pode nos levar a considerar a reestruturação do sistema de formação de quadros que corresponde às necessidades. É importante questionar o quanto o entendimento das metas propostas facilita a criação dos paradigmas corporativos. 

\bibliography{lista_bibliografia}

\apendice

\chapter{Exemplo de Apêndice}

 O incentivo ao avanço tecnológico, assim como a hegemonia do ambiente político cumpre um papel essencial na formulação dos níveis de motivação departamental. Caros amigos, a necessidade de renovação processual pode nos levar a considerar a reestruturação do sistema de formação de quadros que corresponde às necessidades. É importante questionar o quanto o entendimento das metas propostas facilita a criação dos paradigmas corporativos. 
 
\section{Seção de Apêndice}

 O incentivo ao avanço tecnológico, assim como a hegemonia do ambiente político cumpre um papel essencial na formulação dos níveis de motivação departamental. Caros amigos, a necessidade de renovação processual pode nos levar a considerar a reestruturação do sistema de formação de quadros que corresponde às necessidades. É importante questionar o quanto o entendimento das metas propostas facilita a criação dos paradigmas corporativos. 
 
\begin{figure}[h]

\caption{UDESC}
\centering 
\includegraphics[scale=0.04]{fig/Marca_UDESC_vertical.pdf}

\end{figure}


\anexo

\chapter{Exemplo de Anexo}

\begin{figure}

\caption{UDESC}
\centering 
\includegraphics[scale=0.04]{fig/Marca_UDESC_vertical.pdf}

\end{figure}

%%%%%%%%%%%%%%%%%%%%%%%%%%%%%%%%%%%%%%%%%%%%%%%%%%%%%%%%%%%%%%%%%%%%%%
% RESUMO NA ULTIMA PAGINA
%%%%%%%%%%%%%%%%%%%%%%%%%%%%%%%%%%%%%%%%%%%%%%%%%%%%%%%%%%%%%%%%%%%%%%

\pagebreak


\clearpage


\newpage

\thispagestyle{empty}
Parágrafo do resumo
\vspace{.5cm}

\centering Orientador: \orientadordata

\vfill

Joinville, \anodata


\end{document}